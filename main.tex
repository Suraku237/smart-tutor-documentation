\documentclass[12pt,a4paper]{report}
\usepackage[utf8]{inputenc}
\usepackage{amsmath}
\usepackage{amsfonts}
\usepackage{amssymb}
\usepackage{graphicx}
\usepackage{listings}
\usepackage{xcolor}
\usepackage{hyperref}
\usepackage{geometry}
\usepackage{fancyhdr}
\usepackage{longtable}

\geometry{margin=1in}

% --- CODE STYLING ---
\definecolor{codegreen}{rgb}{0,0.6,0}
\definecolor{codegray}{rgb}{0.5,0.5,0.5}
\definecolor{codepurple}{rgb}{0.58,0,0.82}
\definecolor{backcolour}{rgb}{0.95,0.95,0.92}

\lstset{
    backgroundcolor=\color{backcolour},
    commentstyle=\color{codegreen},
    keywordstyle=\color{blue},
    numberstyle=\tiny\color{codegray},
    stringstyle=\color{codepurple},
    basicstyle=\ttfamily\footnotesize,
    breaklines=true,
    captionpos=b,
    keepspaces=true,
    numbers=left,
    showspaces=false,
}

% --- TITLE PAGE ---
\begin{document}

\begin{titlepage}
    \centering
    \vspace*{1cm}
    \huge
    \textbf{Smart Tutor}\\
    \vspace{0.5cm}
    \LARGE
    Comprehensive Software Documentation\\
    (SRS \& SDS)\\
    \vspace{1.5cm}
    \large
    \textbf{Prepared for: Academic Review}\\
    \vfill
\includegraphics[width=0.4\textwidth]{logo_placeholder} logo here
    \vspace{1cm}
    \Large
    January 2026
\end{titlepage}

\tableofcontents
\newpage

% --- CHAPTER 1: SRS ---
\chapter{Software Requirements Specification (SRS)}
\section{Project Introduction}
Smart Tutor is an educational platform designed to facilitate real-time academic discussion and efficient learning resource management. By integrating a Sentiment Hub, students can directly communicate misunderstandings to teachers.

\section{Functional Requirements}
\begin{itemize}
    \item \textbf{Student Module:} Access to PDF resources, real-time inquiry posting in the Sentiment Screen.
    \item \textbf{Admin Module:} Full CRUD (Create, Read, Update, Delete) rights over PDF documents; response management for student sentiments.
\end{itemize}



% --- CHAPTER 2: SDS (Your Design Section) ---
\chapter{Software Design Specification (SDS)}

\section{System Architecture}
The Smart Tutor platform follows a \textbf{Modular Monolith Backend} and a \textbf{Layered Frontend Architecture}. This ensures that the communication logic (Sentiment Hub) does not interfere with the resource management logic (PDF Control).

\subsection{High-Level Component Diagram}
The system is divided into three primary layers:
\begin{enumerate}
    \item \textbf{Presentation Layer (Flutter):} Handles UI rendering, user input validation, and local state management via the Provider pattern.
    \item \textbf{Application Layer (Node.js/Express):} Contains the business logic for message routing, user role verification, and file system interaction.
    \item \textbf{Data Layer (MySQL \& Local SQLite):} Manages persistent storage for global academic data and local user preferences.
\end{enumerate}



\section{Detailed Design: Sentiment Hub}
The Sentiment Hub is designed for low-latency communication. While it currently uses HTTP Polling for stability on mobile networks, the design supports a transition to WebSockets.

\subsection{Message Flow Sequence}
When a student submits a "Misunderstanding Report":
\begin{enumerate}
    \item The Flutter client sends a JSON payload to the \texttt{/api/sentiment/post} endpoint.
    \item The Backend validates the \texttt{userRole} and \texttt{lessonId}.
    \item The record is committed to the MySQL database.
    \item On the next fetch cycle, all participants subscribed to that \texttt{lessonId} receive the update.
\end{enumerate}



\section{Database Design}
\subsection{Entity Relationship Diagram (ERD)}
The relational schema is optimized for fast lookups of learning materials and associated discussions.
\begin{itemize}
    \item \textbf{Users Table:} Stores credentials and roles (Student, Teacher, Admin).
    \item \textbf{Messages Table:} Stores sentiment posts, linked to Users via \texttt{sender\_id}.
    \item \textbf{Resources Table:} Stores metadata for PDF files, including the VPS file path and the difficulty rating.
\end{itemize}



\section{File System Design: Admin PDF Control}
The administration of PDFs follows a "Safe-Deletion" protocol:
\begin{itemize}
    \item \textbf{Upload Path:} Files are moved to \texttt{/var/www/smarttutor/resources/}.
    \item \textbf{Naming Convention:} Files utilize UUIDs to prevent filename collisions.
    \item \textbf{Deletion:} Uses \texttt{fs.unlink()} to physically remove file blocks from the VPS storage.
\end{itemize}

\section{DevOps \& Deployment Design}
\subsection{Infrastructure Stack}
\begin{table}[h]
\centering
\begin{tabular}{|l|l|}
\hline
\textbf{Component} & \textbf{Technology} \\ \hline
Operating System & Ubuntu 22.04 LTS \\ \hline
Reverse Proxy & Nginx \\ \hline
Process Manager & PM2 \\ \hline
Database & MySQL 8.0 \\ \hline
\end{tabular}
\caption{System Infrastructure Stack}
\end{table}

\subsection{Deployment Workflow}
The DevOps pipeline utilizes Git. The process includes a pre-compilation step where Flutter assets are optimized for production.



\section{User Interface Design Principles}
UI design adheres to \textbf{Material Design 3} standards, featuring accessibility focus and loading states for data-heavy PDF actions.

\chapter{Project Methodology (Scrum)}
\section{Sprint Cycles}
The project was developed in three 2-week sprints, focusing on user authentication, the sentiment engine, and finally the admin resource tools.
% --- CHAPTER 4: AGILE PROJECT MANAGEMENT ---
\chapter{Agile Implementation: Scrum Framework}

\section{Product Backlog}
The Product Backlog represents the master list of all functional requirements and features desired for the Smart Tutor platform. These items are prioritized based on their value to the student-teacher collaboration loop.

\begin{longtable}{|p{2cm}|p{6cm}|p{3cm}|p{2cm}|}
\hline
\textbf{ID} & \textbf{User Story} & \textbf{Priority} & \textbf{Status} \\ \hline
PB-01 & As a user, I want to sign up and log in securely. & High & Completed \\ \hline
PB-02 & As a student, I want to post academic inquiries (Sentiments) so teachers can help me. & High & Completed \\ \hline
PB-03 & As a teacher, I want to respond to student inquiries in real-time. & High & Completed \\ \hline
PB-04 & As an admin, I want to upload PDF study materials to the VPS. & Medium & Completed \\ \hline
PB-05 & As an admin, I want to delete difficult or outdated PDFs to keep the library clean. & Medium & Completed \\ \hline
PB-06 & As a user, I want to upload a local profile picture for identification. & Low & Completed \\ \hline
PB-07 & As a user, I want a dark mode interface for night-time studying. & Low & Completed \\ \hline
\end{longtable}

\section{Sprint Schedule and Execution}
The project was executed over three distinct Sprints, each lasting 14 days (2 weeks).

\subsection{Sprint 1: Core Authentication \& Infrastructure}
\textbf{Duration:} Day 1 - Day 14 \\
\textbf{Goal:} Establish the backend API and basic user access.
\begin{itemize}
    \item \textbf{Tasks:} 
    \begin{enumerate}
        \item Setup Ubuntu VPS with Nginx and MySQL.
        \item Implement JWT-based authentication in Node.js.
        \item Create Flutter Login and Signup UI.
        \item Integrate local SQLite for profile path storage.
    \end{enumerate}
    \item \textbf{Deliverable:} A working login system that identifies User Roles (Admin vs Student).
\end{itemize}



\subsection{Sprint 2: Sentiment Hub Development}
\textbf{Duration:} Day 15 - Day 28 \\
\textbf{Goal:} Enable student-teacher communication.
\begin{itemize}
    \item \textbf{Tasks:} 
    \begin{enumerate}
        \item Create the Sentiment database schema (Messages table).
        \item Develop the Flutter Sentiment Screen with message bubbles.
        \item Implement API polling to fetch new messages from the server.
        \item Add "Lesson Context" tagging to inquiries.
    \end{enumerate}
    \item \textbf{Deliverable:} A functional community chat hub for academic support.
\end{itemize}



\subsection{Sprint 3: Admin Resource Management \& DevOps}
\textbf{Duration:} Day 29 - Day 42 \\
\textbf{Goal:} Empower admins to manage learning materials.
\begin{itemize}
    \item \textbf{Tasks:} 
    \begin{enumerate}
        \item Implement File Upload logic using \texttt{multer} on the backend.
        \item Create the Admin PDF Management dashboard in Flutter.
        \item Script the \texttt{fs.unlink} process for safe PDF deletion.
        \item Finalize DevOps deployment using PM2 for process monitoring.
    \end{enumerate}
    \item \textbf{Deliverable:} A full-stack system allowing dynamic PDF curation.
\end{itemize}



\chapter{DevOps and Maintenance}
\section{Process Management}
To ensure the "Smart Tutor" backend remains online 24/7, we utilized **PM2 (Process Manager 2)**. This allows the API to automatically restart if a runtime error occurs or if the VPS undergoes a reboot.

\section{Security Protocols}
All API endpoints are protected by role-based middleware. A student cannot access the \texttt{/admin/upload} endpoint; the server validates the user's role stored in the database before processing any file system changes.

\chapter{Conclusion}
The Smart Tutor project demonstrates the power of collaborative software design. By removing automated session management and focusing on human sentiment and curated resources, the platform provides a superior educational experience. The Scrum framework ensured that all high-priority features were delivered within the 6-week development window.

\end{document}
\end{document}