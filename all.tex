\documentclass[12pt,a4paper]{report}
\usepackage[utf8]{inputenc}
\usepackage{listings}
\usepackage{xcolor}
\usepackage{geometry}
\usepackage{hyperref}

\geometry{margin=1in}

% --- COLOR SCHEME FOR CODE ---
\definecolor{commentgreen}{rgb}{0,0.6,0}
\definecolor{keywordblue}{rgb}{0,0,0.8}
\definecolor{stringorange}{rgb}{0.8,0.4,0}
\definecolor{backcolor}{rgb}{0.97,0.97,0.97}

\lstset{
    backgroundcolor=\color{backcolor},
    commentstyle=\color{commentgreen},
    keywordstyle=\color{keywordblue},
    stringstyle=\color{stringorange},
    basicstyle=\ttfamily\small,
    breaklines=true,
    captionpos=b,
    numbers=left,
    numberstyle=\tiny,
    frame=single,
    showstringspaces=false
}

\begin{document}

\title{\huge \textbf{Smart Tutor}\\ \Large Technical Implementation \& Codebase Reference}
\author{Lead Developer}
\date{January 2026}
\maketitle

\tableofcontents
\newpage

\chapter{Frontend: Flutter Implementation}

\section{App Entry and Routing}
The application uses a centralized routing system. This ensures that the Admin and Student views are kept separate from the moment of authentication.

\begin{lstlisting}[language=SQL, caption=Main Entry Point (main.dart)]
void main() {
  runApp(
    MultiProvider(
      providers: [
        ChangeNotifierProvider(create: (_) => AuthProvider()),
        ChangeNotifierProvider(create: (_) => SentimentProvider()),
      ],
      child: SmartTutorApp(),
    ),
  );
}
\end{lstlisting}

\section{The Sentiment Hub UI}
The Sentiment Hub uses a `ListView.builder` to render real-time messages. It filters messages based on the \texttt{lessonId}.



\begin{lstlisting}[language=SQL, caption=Sentiment Screen Logic]
class SentimentScreen extends StatelessWidget {
  @override
  Widget build(BuildContext context) {
    return Scaffold(
      appBar: AppBar(title: Text("Academic Sentiment Hub")),
      body: Consumer<SentimentProvider>(
        builder: (context, provider, child) {
          return ListView.builder(
            itemCount: provider.messages.length,
            itemBuilder: (context, index) {
              return ChatBubble(
                message: provider.messages[index].text,
                isStudent: provider.messages[index].role == 'student',
              );
            },
          );
        },
      ),
    );
  }
}
\end{lstlisting}

\chapter{Backend: Node.js \& Express API}

\section{Database Connection Management}
The backend connects to a MySQL database hosted on the Linux VPS. We use a connection pool to handle multiple simultaneous student requests.

\begin{lstlisting}[language=SQL, caption=Database Configuration (db.js)]
const mysql = require('mysql2');

const pool = mysql.createPool({
  host: 'your-vps-ip',
  user: 'admin',
  password: 'securepassword',
  database: 'smart_tutor',
  waitForConnections: true,
  connectionLimit: 10
});

module.exports = pool.promise();
\end{lstlisting}

\section{Resource Management Logic}
This is the core "Admin" feature. It handles the physical deletion of files from the VPS storage using the \texttt{fs} module.



\begin{lstlisting}[language=SQL, caption=PDF Deletion Controller (resourceController.js)]
const fs = require('fs');
const path = require('path');

exports.deleteResource = async (req, res) => {
  const { fileId, fileName } = req.body;
  const filePath = path.join(__dirname, '../resources/', fileName);

  try {
    // 1. Remove from Database
    await db.query('DELETE FROM resources WHERE id = ?', [fileId]);

    // 2. Remove physical file from VPS disk
    if (fs.existsSync(filePath)) {
      fs.unlinkSync(filePath);
      res.status(200).send("Resource deleted successfully");
    }
  } catch (err) {
    res.status(500).send("Error during deletion: " + err.message);
  }
};
\end{lstlisting}

\chapter{Data Layer: Schema Design}

\section{Relational Tables}
The following SQL represents the production schema for the Smart Tutor database.



\begin{lstlisting}[language=SQL, caption=MySQL Production Schema]
CREATE TABLE users (
    id INT PRIMARY KEY AUTO_INCREMENT,
    email VARCHAR(255) UNIQUE,
    password_hash VARCHAR(255),
    role ENUM('student', 'admin') DEFAULT 'student'
);

CREATE TABLE sentiment_logs (
    id INT PRIMARY KEY AUTO_INCREMENT,
    user_id INT,
    message TEXT,
    created_at TIMESTAMP DEFAULT CURRENT_TIMESTAMP,
    FOREIGN KEY (user_id) REFERENCES users(id)
);
\end{lstlisting}

\chapter{Conclusion}
The combination of Flutter's reactive UI and Node.js's efficient file handling allows for a high-performance educational tool. By utilizing the VPS storage directly, we ensure that students always have access to the latest curated PDF materials.

\end{document}