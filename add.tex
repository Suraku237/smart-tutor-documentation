\documentclass[12pt,a4paper]{report}
\usepackage[utf8]{inputenc}
\usepackage{amsmath}
\usepackage{amsfonts}
\usepackage{amssymb}
\usepackage{graphicx}
\usepackage{longtable}
\usepackage{geometry}
\usepackage{listings}
\usepackage{xcolor}

\geometry{margin=1in}

\begin{document}

% --- CHAPTER 2: SOFTWARE DESIGN SPECIFICATION ---
\chapter{Software Design Specification (SDS)}

\section{System Architecture}
The Smart Tutor platform is designed using a \textbf{Three-Tier Architecture} model. This ensures a clean separation of concerns, allowing the mobile frontend to evolve independently of the backend logic and the database storage.



\subsection{Presentation Layer (Frontend)}
Developed using the \textbf{Flutter Framework}, this layer manages user interactions. It utilizes the \textbf{Provider Pattern} for state management, ensuring that the UI reacts dynamically to data changes in the Sentiment Hub without requiring full-page reloads.

\subsection{Application Layer (Logic)}
The backend is a \textbf{Node.js and Express} REST API. This layer handles business logic, including user authentication (JWT), file system management for PDF resources on the VPS, and message routing for the Sentiment Hub.

\subsection{Data Layer (Persistence)}
This layer consists of a \textbf{MySQL Relational Database} for structured data (users, messages) and a \textbf{Local File System} on the Ubuntu VPS for unstructured data (PDF documents).



\section{Requirements Modeling}
To define the system's behavior, we utilize Use Case Modeling to represent the interactions between actors and the system boundaries.

\subsection{Use Case Diagram}
The Use Case Diagram illustrates the functional requirements from the perspective of the two primary actors: the \textbf{Student} and the \textbf{Administrator}.



\subsection{Use Case Descriptions}
\begin{itemize}
    \item \textbf{Upload Resource (Admin):} The Admin selects a PDF. The system validates the file type, renames it using a UUID to prevent collisions, and stores the path in the database.
    \item \textbf{Post Sentiment (Student):} The Student submits an inquiry. The system captures the lesson context and stores the message for the Teacher to review.
\end{itemize}

\section{System Structural Design}
\subsection{Class Diagram}
The Class Diagram represents the static structure of the system. It defines the entities, their attributes, and how they relate to one another (Composition, Association, and Inheritance).



\subsection{Object Relationships}
\begin{itemize}
    \item \textbf{User to Message:} A One-to-Many relationship; one user can post multiple sentiment messages.
    \item \textbf{Message to Lesson:} A Many-to-One relationship; many inquiries can be linked to a single academic lesson.
\end{itemize}

\section{Data Design}
\subsection{Entity Relationship Diagram (ERD)}
The database schema is designed to ensure data integrity and optimized query performance for real-time chat fetching.



\subsection{Data Dictionary}
\begin{longtable}{|p{3cm}|p{2cm}|p{8cm}|}
\hline
\textbf{Attribute} & \textbf{Data Type} & \textbf{Description} \\ \hline
\texttt{user\_id} & INT (PK) & Unique identifier for the user profile. \\ \hline
\texttt{file\_path} & VARCHAR & The physical directory location of the PDF on the VPS. \\ \hline
\texttt{created\_at} & TIMESTAMP & Automatic record of the time of entry. \\ \hline
\end{longtable}

\section{Interface Design and UX Flow}
The user experience is designed around a \textbf{Linear Navigation Flow}. Users must undergo a state-check (Authentication) before accessing the core modules of the app.



\chapter{Project Implementation Details}
\section{Coding Standards}
The project adheres to the \textbf{SOLID principles} of object-oriented design. For instance, the File Service is decoupled from the Controller logic, allowing the Admin to delete PDFs without affecting the database uptime.

\end{document}