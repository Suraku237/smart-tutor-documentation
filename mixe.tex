\documentclass[12pt,a4paper]{report}
\usepackage[utf8]{inputenc}
\usepackage{amsmath}
\usepackage{amsfonts}
\usepackage{amssymb}
\usepackage{graphicx}
\usepackage{longtable}
\usepackage{geometry}
\usepackage{listings}
\usepackage{xcolor}
\usepackage{hyperref}
\usepackage{fancyhdr}
\usepackage{tocloft}
\usepackage{float}
\usepackage{caption}
\usepackage{subcaption}

\geometry{margin=1in}

% --- CODE STYLING ---
\definecolor{codegreen}{rgb}{0,0.6,0}
\definecolor{codegray}{rgb}{0.5,0.5,0.5}
\definecolor{codepurple}{rgb}{0.58,0,0.82}
\definecolor{backcolour}{rgb}{0.95,0.95,0.92}
\definecolor{commentgreen}{rgb}{0,0.6,0}
\definecolor{keywordblue}{rgb}{0,0,0.8}
\definecolor{stringorange}{rgb}{0.8,0.4,0}
\definecolor{backcolor}{rgb}{0.97,0.97,0.97}

% First lstset for general code
\lstset{
    backgroundcolor=\color{backcolour},
    commentstyle=\color{codegreen},
    keywordstyle=\color{blue},
    numberstyle=\tiny\color{codegray},
    stringstyle=\color{codepurple},
    basicstyle=\ttfamily\footnotesize,
    breaklines=true,
    captionpos=b,
    keepspaces=true,
    numbers=left,
    showspaces=false,
}

% Second lstset for SQL highlighting
\lstdefinestyle{sqlstyle}{
    backgroundcolor=\color{backcolor},
    commentstyle=\color{commentgreen},
    keywordstyle=\color{keywordblue},
    stringstyle=\color{stringorange},
    basicstyle=\ttfamily\small,
    breaklines=true,
    captionpos=b,
    numbers=left,
    numberstyle=\tiny,
    frame=single,
    showstringspaces=false,
    language=SQL
}

% --- HEADER AND FOOTER ---
\pagestyle{fancy}
\fancyhf{}
\fancyhead[L]{\leftmark}
\fancyfoot[C]{\thepage}
\renewcommand{\headrulewidth}{0.4pt}
\renewcommand{\footrulewidth}{0.4pt}

% --- TITLE PAGE ---
\begin{document}

\begin{titlepage}
    \centering
    \vspace*{1cm}
    \huge
    \textbf{Smart Tutor - Comprehensive Documentation}\\
    \vspace{0.5cm}
    \LARGE
    Software Requirements Specification (SRS) \&\\
    Software Design Specification (SDS)\\
    \vspace{1.5cm}
    \large
    \textbf{A Complete Technical Implementation Guide}\\
    \vspace{0.5cm}
    \normalsize
    Prepared for Academic Review and Development Reference\\
    \vspace{2cm}
    
    % Placeholder for logo - replace with actual logo file
    \IfFileExists{logo.png}{
        \includegraphics[width=0.4\textwidth]{logo.png}
    }{
        \fbox{\parbox{0.4\textwidth}{\centering Logo Here\\[2cm]}}
    }
    
    \vfill
    \Large
    \textbf{Version 1.0}\\
    \vspace{0.5cm}
    \large
    January 2026
\end{titlepage}

% --- COPYRIGHT PAGE ---
\newpage
\thispagestyle{empty}
\vspace*{\fill}
\begin{center}
    \copyright\ 2026 Smart Tutor Development Team\\
    \vspace{1cm}
    This document contains proprietary information of the Smart Tutor project.\\
    Distribution is restricted to authorized personnel only.
\end{center}
\vspace*{\fill}

% --- TABLE OF CONTENTS ---
\tableofcontents
\newpage

% --- LIST OF FIGURES ---
\listoffigures
\newpage

% --- LIST OF TABLES ---
\listoftables
\newpage

% ============================================
% CHAPTER 1: INTRODUCTION & SRS
% ============================================
\chapter{Software Requirements Specification (SRS)}
\label{ch:srs}

\section{Project Introduction}
Smart Tutor is an innovative educational platform designed to bridge the gap between students and educators through real-time academic discussions and efficient learning resource management. The platform integrates a unique "Sentiment Hub" feature that allows students to directly communicate misunderstandings to teachers, fostering a more responsive and personalized learning environment.

\section{Vision Statement}
To create a digital learning ecosystem where students can actively voice academic challenges and receive timely support, while administrators can efficiently curate and manage high-quality learning resources.

\section{Scope}
The Smart Tutor platform encompasses:
\begin{itemize}
    \item A mobile application built with Flutter for iOS and Android
    \item A backend REST API built with Node.js and Express
    \item A MySQL database for structured data storage
    \item VPS-based file storage for learning resources (PDFs)
    \item Real-time communication capabilities through the Sentiment Hub
\end{itemize}

\section{Functional Requirements}
\subsection{Student Module}
\begin{longtable}{|p{3cm}|p{10cm}|}
\hline
\textbf{Requirement ID} & \textbf{Description} \\ \hline
FR-ST-01 & Students shall be able to register and authenticate using email and password \\ \hline
FR-ST-02 & Students shall have access to PDF learning resources organized by lessons \\ \hline
FR-ST-03 & Students shall be able to post real-time academic inquiries in the Sentiment Hub \\ \hline
FR-ST-04 & Students shall view responses from teachers/administrators \\ \hline
FR-ST-05 & Students shall have a personalized dashboard showing their activity \\ \hline
\end{longtable}

\subsection{Administrator Module}
\begin{longtable}{|p{3cm}|p{10cm}|}
\hline
\textbf{Requirement ID} & \textbf{Description} \\ \hline
FR-AD-01 & Administrators shall have full CRUD (Create, Read, Update, Delete) rights over PDF documents \\ \hline
FR-AD-02 & Administrators shall be able to manage user accounts and roles \\ \hline
FR-AD-03 & Administrators shall respond to student sentiments in the Sentiment Hub \\ \hline
FR-AD-04 & Administrators shall monitor system usage and analytics \\ \hline
FR-AD-05 & Administrators shall manage lesson structures and course content \\ \hline
\end{longtable}

\section{Non-Functional Requirements}
\begin{longtable}{|p{3cm}|p{10cm}|}
\hline
\textbf{Category} & \textbf{Requirement} \\ \hline
Performance & System shall support up to 1000 concurrent users with response time under 2 seconds \\ \hline
Availability & System shall achieve 99.5\% uptime excluding scheduled maintenance \\ \hline
Security & All user data shall be encrypted in transit (TLS 1.2+) and at rest \\ \hline
Scalability & Architecture shall support horizontal scaling of backend services \\ \hline
Usability & Mobile app shall follow Material Design 3 guidelines with accessibility support \\ \hline
\end{longtable}

% ============================================
% CHAPTER 2: SYSTEM ARCHITECTURE
% ============================================
\chapter{System Architecture Design}
\label{ch:architecture}

\section{Architectural Overview}
The Smart Tutor platform employs a \textbf{Three-Tier Architecture} model, ensuring clear separation of concerns between presentation, application logic, and data persistence layers. This architecture enables independent evolution of mobile frontend, backend services, and database systems.

\begin{figure}[H]
    \centering
    \includegraphics[width=0.8\textwidth]{architecture-diagram.png}
    \caption{Three-Tier Architecture of Smart Tutor Platform}
    \label{fig:architecture}
\end{figure}

\subsection{Presentation Layer (Frontend)}
\begin{itemize}
    \item \textbf{Technology:} Flutter Framework (Dart)
    \item \textbf{State Management:} Provider Pattern
    \item \textbf{Key Features:}
    \begin{itemize}
        \item Responsive UI for iOS and Android
        \item Real-time UI updates via reactive programming
        \item Offline capability for cached resources
        \item Accessibility features compliant with WCAG 2.1
    \end{itemize}
\end{itemize}

\subsection{Application Layer (Backend)}
\begin{itemize}
    \item \textbf{Technology:} Node.js with Express.js
    \item \textbf{Key Components:}
    \begin{itemize}
        \item RESTful API endpoints
        \item JWT-based authentication middleware
        \role-based access control (RBAC)
        \item File upload/download handlers
        \item Real-time message routing (polling with WebSocket readiness)
    \end{itemize}
    \item \textbf{Pattern:} MVC (Model-View-Controller) architecture
\end{itemize}

\subsection{Data Layer (Persistence)}
\begin{itemize}
    \item \textbf{Structured Data:} MySQL 8.0 Relational Database
    \begin{itemize}
        \item User management and authentication
        \item Message logging and sentiment tracking
        \item Resource metadata and relationships
    \end{itemize}
    \item \textbf{Unstructured Data:} Local File System on Ubuntu VPS
    \begin{itemize}
        \item PDF document storage
        \item User profile pictures
        \item Temporary upload files
    \end{itemize}
\end{itemize}

\section{Component Diagram}
The system components interact as shown in Figure \ref{fig:components}:

\begin{figure}[H]
    \centering
    \includegraphics[width=\textwidth]{component-diagram.png}
    \caption{System Component Interaction Diagram}
    \label{fig:components}
\end{figure}

% ============================================
% CHAPTER 3: REQUIREMENTS MODELING
% ============================================
\chapter{Requirements Modeling}
\label{ch:requirements-modeling}

\section{Use Case Modeling}
Use Case modeling defines the system's behavior from the perspective of primary actors, establishing clear boundaries for system functionality.

\subsection{Use Case Diagram}
Figure \ref{fig:usecase} illustrates the functional requirements and interactions between the primary actors (Student and Administrator) and the system.

\begin{figure}[H]
    \centering
    \includegraphics[width=0.9\textwidth]{use-case-diagram.png}
    \caption{Smart Tutor Use Case Diagram}
    \label{fig:usecase}
\end{figure}

\subsection{Actor Definitions}
\begin{longtable}{|p{4cm}|p{10cm}|}
\hline
\textbf{Actor} & \textbf{Description} \\ \hline
Student & Primary consumer of learning resources and sender of academic inquiries \\ \hline
Administrator & System manager responsible for content curation and user support \\ \hline
System & Automated processes including notifications, backups, and analytics \\ \hline
\end{longtable}

\subsection{Detailed Use Case Specifications}

\subsubsection{Use Case UC-01: Post Academic Sentiment}
\begin{longtable}{|p{4cm}|p{10cm}|}
\hline
\textbf{Use Case ID} & UC-01 \\ \hline
\textbf{Name} & Post Academic Sentiment \\ \hline
\textbf{Primary Actor} & Student \\ \hline
\textbf{Secondary Actor} & System \\ \hline
\textbf{Description} & Allows a student to submit a real-time report regarding confusion or misunderstanding of a specific lesson \\ \hline
\textbf{Pre-conditions} & Student is authenticated; Valid Lesson ID exists \\ \hline
\textbf{Main Flow} & 
1. Student navigates to Sentiment Hub \\
2. Student selects relevant lesson context \\
3. Student types inquiry message \\
4. System validates input (length, content) \\
5. System stores message in database with timestamp \\
6. System notifies subscribed administrators \\
7. System confirms successful post to student \\ \hline
\textbf{Post-conditions} & Sentiment is stored and visible to authorized users \\ \hline
\textbf{Exceptions} & Network failure, Invalid lesson ID, Content violation \\ \hline
\end{longtable}

\subsubsection{Use Case UC-02: Manage PDF Resources}
\begin{longtable}{|p{4cm}|p{10cm}|}
\hline
\textbf{Use Case ID} & UC-02 \\ \hline
\textbf{Name} & Manage PDF Resources (CRUD) \\ \hline
\textbf{Primary Actor} & Administrator \\ \hline
\textbf{Description} & Enables admin to upload new PDFs or delete outdated materials from VPS storage \\ \hline
\textbf{Pre-conditions} & Admin is authenticated; Has resource management privileges \\ \hline
\textbf{Main Flow} & 
1. Admin navigates to Resource Dashboard \\
2. Admin selects 'Delete' on target resource \\
3. System prompts for confirmation \\
4. Admin confirms deletion \\
5. System removes database entry \\
6. System executes \texttt{fs.unlink()} on VPS file path \\
7. System logs deletion activity \\
8. System updates UI to reflect removal \\ \hline
\textbf{Post-conditions} & Resource is permanently removed from system \\ \hline
\textbf{Exceptions} & File not found, Permission denied, Database constraint violation \\ \hline
\end{longtable}

\subsubsection{Use Case UC-03: Upload Learning Resource}
\begin{longtable}{|p{4cm}|p{10cm}|}
\hline
\textbf{Use Case ID} & UC-03 \\ \hline
\textbf{Name} & Upload Learning Resource \\ \hline
\textbf{Primary Actor} & Administrator \\ \hline
\textbf{Description} & Admin uploads a new PDF learning material to the VPS storage \\ \hline
\textbf{Pre-conditions} & Admin authenticated; Valid PDF file available \\ \hline
\textbf{Main Flow} & 
1. Admin selects PDF file from device \\
2. System validates file type and size \\
3. System generates UUID filename \\
4. System stores file in \texttt{/var/www/smarttutor/resources/} \\
5. System creates database record with metadata \\
6. System confirms successful upload \\ \hline
\textbf{Post-conditions} & Resource available to students \\ \hline
\end{longtable}

% ============================================
% CHAPTER 4: STRUCTURAL DESIGN
% ============================================
\chapter{Structural Design}
\label{ch:structural-design}

\section{Class Diagram}
The Class Diagram represents the static structure of the Smart Tutor system, defining entities, their attributes, methods, and relationships.

\begin{figure}[H]
    \centering
    \includegraphics[width=\textwidth]{Class-Diagram-of-Online-Learning-System.png}
    \caption{Smart Tutor Class Diagram}
    \label{fig:class-diagram}
\end{figure}

\subsection{Key Classes and Relationships}

\subsubsection{User Management Classes}
\begin{itemize}
    \item \textbf{Users:} Base class for all system users
    \begin{itemize}
        \item Attributes: Username, Email, Password (hashed), Role, Registration\_Date
        \item Methods: getUser(), addUser(), editUser(), authenticate()
    \end{itemize}
    \item \textbf{Admin:} Specialized user with system management privileges
    \begin{itemize}
        \item Methods: manageAssessment(), manageCourse(), manageUser(), etc.
    \end{itemize}
\end{itemize}

\subsubsection{Content Management Classes}
\begin{itemize}
    \item \textbf{Course\_Group:} Organizes related learning materials
    \item \textbf{Topic:} Represents major subject areas within courses
    \item \textbf{Lesson:} Individual learning units with specific content
    \item \textbf{Subtopic:} Detailed breakdown of lesson components
\end{itemize}

\subsubsection{Interaction Classes}
\begin{itemize}
    \item \textbf{Thread:} Discussion threads for academic conversations
    \item \textbf{Comment:} Individual messages within threads
    \item \textbf{Material\_Support:} Learning resources (PDFs, videos, etc.)
    \item \textbf{Test\_Transaction:} Track student assessment activities
\end{itemize}

\subsection{Relationship Types}
\begin{longtable}{|p{4cm}|p{10cm}|}
\hline
\textbf{Relationship} & \textbf{Description} \\ \hline
Association & General "uses" relationship between classes \\ \hline
Composition & Strong ownership (whole-part) relationship \\ \hline
Aggregation & Weak ownership relationship \\ \hline
Inheritance & "is-a" relationship for specialization \\ \hline
Dependency & One class depends on another \\ \hline
\end{longtable}

\section{Object Relationships}
\begin{itemize}
    \item \textbf{User to Message:} One-to-Many relationship (one user can post multiple sentiment messages)
    \item \textbf{Message to Lesson:} Many-to-One relationship (multiple messages can reference a single lesson)
    \item \textbf{Admin to Resource:} One-to-Many relationship (one admin can manage multiple resources)
    \item \textbf{Student to Test\_Transaction:} One-to-Many relationship (one student can have multiple test attempts)
\end{itemize}

% ============================================
% CHAPTER 5: DATA DESIGN
% ============================================
\chapter{Data Design}
\label{ch:data-design}

\section{Entity Relationship Diagram (ERD)}
The ERD illustrates the logical structure of the database, showing entities, attributes, and relationships.

\begin{figure}[H]
    \centering
    \includegraphics[width=\textwidth]{entity-relationship-diagram.png}
    \caption{Smart Tutor Entity Relationship Diagram}
    \label{fig:erd}
\end{figure}

\section{Database Schema}
\subsection{Core Tables Design}

\begin{lstlisting}[style=sqlstyle, caption=MySQL Production Schema]
-- Users Table: Stores authentication and profile data
CREATE TABLE users (
    id INT PRIMARY KEY AUTO_INCREMENT,
    email VARCHAR(255) UNIQUE NOT NULL,
    password_hash VARCHAR(255) NOT NULL,
    first_name VARCHAR(100),
    last_name VARCHAR(100),
    role ENUM('student', 'teacher', 'admin') DEFAULT 'student',
    profile_picture_path VARCHAR(500),
    registration_date TIMESTAMP DEFAULT CURRENT_TIMESTAMP,
    last_login TIMESTAMP NULL,
    status ENUM('active', 'inactive', 'suspended') DEFAULT 'active',
    INDEX idx_email (email),
    INDEX idx_role (role)
) ENGINE=InnoDB DEFAULT CHARSET=utf8mb4 COLLATE=utf8mb4_unicode_ci;

-- Sentiment Logs Table: Stores academic inquiries
CREATE TABLE sentiment_logs (
    id INT PRIMARY KEY AUTO_INCREMENT,
    user_id INT NOT NULL,
    lesson_id INT NOT NULL,
    message TEXT NOT NULL,
    sentiment_type ENUM('confusion', 'question', 'feedback', 'suggestion'),
    priority ENUM('low', 'medium', 'high') DEFAULT 'medium',
    status ENUM('pending', 'answered', 'archived') DEFAULT 'pending',
    created_at TIMESTAMP DEFAULT CURRENT_TIMESTAMP,
    updated_at TIMESTAMP DEFAULT CURRENT_TIMESTAMP ON UPDATE CURRENT_TIMESTAMP,
    FOREIGN KEY (user_id) REFERENCES users(id) ON DELETE CASCADE,
    FOREIGN KEY (lesson_id) REFERENCES lessons(id) ON DELETE CASCADE,
    INDEX idx_user_lesson (user_id, lesson_id),
    INDEX idx_status (status),
    INDEX idx_created (created_at)
) ENGINE=InnoDB DEFAULT CHARSET=utf8mb4 COLLATE=utf8mb4_unicode_ci;

-- Resources Table: Manages learning materials
CREATE TABLE resources (
    id INT PRIMARY KEY AUTO_INCREMENT,
    file_uuid VARCHAR(36) UNIQUE NOT NULL,
    original_filename VARCHAR(255) NOT NULL,
    file_path VARCHAR(500) NOT NULL,
    file_size BIGINT,
    mime_type VARCHAR(100),
    lesson_id INT,
    difficulty ENUM('beginner', 'intermediate', 'advanced'),
    uploader_id INT NOT NULL,
    upload_date TIMESTAMP DEFAULT CURRENT_TIMESTAMP,
    last_accessed TIMESTAMP NULL,
    access_count INT DEFAULT 0,
    is_active BOOLEAN DEFAULT TRUE,
    FOREIGN KEY (lesson_id) REFERENCES lessons(id) ON DELETE SET NULL,
    FOREIGN KEY (uploader_id) REFERENCES users(id) ON DELETE CASCADE,
    INDEX idx_uuid (file_uuid),
    INDEX idx_lesson (lesson_id),
    INDEX idx_uploader (uploader_id)
) ENGINE=InnoDB DEFAULT CHARSET=utf8mb4 COLLATE=utf8mb4_unicode_ci;

-- Lessons Table: Course content structure
CREATE TABLE lessons (
    id INT PRIMARY KEY AUTO_INCREMENT,
    title VARCHAR(255) NOT NULL,
    description TEXT,
    content LONGTEXT,
    lesson_number INT,
    topic_id INT,
    created_by INT,
    created_at TIMESTAMP DEFAULT CURRENT_TIMESTAMP,
    updated_at TIMESTAMP DEFAULT CURRENT_TIMESTAMP ON UPDATE CURRENT_TIMESTAMP,
    FOREIGN KEY (topic_id) REFERENCES topics(id) ON DELETE CASCADE,
    FOREIGN KEY (created_by) REFERENCES users(id) ON DELETE SET NULL,
    INDEX idx_topic (topic_id),
    INDEX idx_created (created_at)
) ENGINE=InnoDB DEFAULT CHARSET=utf8mb4 COLLATE=utf8mb4_unicode_ci;
\end{lstlisting}

\section{Data Dictionary}
\begin{longtable}{|p{3cm}|p{2.5cm}|p{9cm}|}
\hline
\textbf{Attribute} & \textbf{Data Type} & \textbf{Description} \\ \hline
\texttt{user\_id} & INT (PK) & Unique identifier for user profiles, auto-incremented \\ \hline
\texttt{file\_uuid} & VARCHAR(36) & Universally Unique Identifier for file tracking, prevents naming conflicts \\ \hline
\texttt{file\_path} & VARCHAR(500) & Physical directory location on VPS (e.g., /var/www/smarttutor/resources/) \\ \hline
\texttt{password\_hash} & VARCHAR(255) & BCrypt hashed password for secure authentication \\ \hline
\texttt{role} & ENUM & Defines access level: 'student', 'teacher', or 'admin' \\ \hline
\texttt{sentiment\_type} & ENUM & Categorizes inquiry: 'confusion', 'question', 'feedback', 'suggestion' \\ \hline
\texttt{priority} & ENUM & Message urgency: 'low', 'medium', 'high' \\ \hline
\texttt{created\_at} & TIMESTAMP & Automatic record creation timestamp \\ \hline
\texttt{updated\_at} & TIMESTAMP & Last modification timestamp, auto-updated \\ \hline
\texttt{is\_active} & BOOLEAN & Soft delete flag for resources and users \\ \hline
\texttt{access\_count} & INT & Tracks resource popularity and usage \\ \hline
\end{longtable}

\section{Indexing Strategy}
\begin{itemize}
    \item \textbf{Primary Keys:} All tables have auto-incrementing INT primary keys
    \item \textbf{Foreign Keys:} Proper indexing on all foreign key columns
    \item \textbf{Composite Indexes:} Frequently queried combinations (user\_id + lesson\_id)
    \item \textbf{Full-text Search:} Implemented on message content for sentiment searching
\end{itemize}

% ============================================
% CHAPTER 6: BEHAVIORAL DESIGN
% ============================================
\chapter{Behavioral Design}
\label{ch:behavioral-design}

\section{Sequence Diagrams}

\subsection{Post Sentiment Sequence}
Figure \ref{fig:seq-sentiment} illustrates the message flow when a student posts an academic inquiry.

\begin{figure}[H]
    \centering
    \includegraphics[width=0.9\textwidth]{sequence-sentiment-post.png}
    \caption{Sequence Diagram: Posting Academic Sentiment}
    \label{fig:seq-sentiment}
\end{figure}

\subsection{PDF Deletion Sequence}
Figure \ref{fig:seq-delete} shows the secure deletion process for PDF resources.

\begin{figure}[H]
    \centering
    \includegraphics[width=0.9\textwidth]{sequence-pdf-delete.png}
    \caption{Sequence Diagram: Admin PDF Deletion}
    \label{fig:seq-delete}
\end{figure}

\section{State Diagrams}

\subsection{User Authentication States}
\begin{figure}[H]
    \centering
    \includegraphics[width=0.6\textwidth]{state-auth.png}
    \caption{State Diagram: User Authentication Flow}
    \label{fig:state-auth}
\end{figure}

\subsection{Message Lifecycle States}
\begin{figure}[H]
    \centering
    \includegraphics[width=0.7\textwidth]{state-message.png}
    \caption{State Diagram: Sentiment Message Lifecycle}
    \label{fig:state-message}
\end{figure}

% ============================================
% CHAPTER 7: FRONTEND IMPLEMENTATION
% ============================================
\chapter{Frontend Implementation: Flutter}
\label{ch:frontend}

\section{Architecture Overview}
The Flutter frontend follows the MVVM (Model-View-ViewModel) pattern with Provider for state management, ensuring clean separation between UI, business logic, and data layers.

\section{App Entry and Routing}
\begin{lstlisting}[style=sqlstyle, caption=Main Entry Point (main.dart)]
import 'package:flutter/material.dart';
import 'package:provider/provider.dart';
import 'package:smart_tutor/providers/auth_provider.dart';
import 'package:smart_tutor/providers/sentiment_provider.dart';
import 'package:smart_tutor/providers/resource_provider.dart';
import 'package:smart_tutor/screens/splash_screen.dart';

void main() {
  runApp(
    MultiProvider(
      providers: [
        ChangeNotifierProvider(create: (_) => AuthProvider()),
        ChangeNotifierProvider(create: (_) => SentimentProvider()),
        ChangeNotifierProvider(create: (_) => ResourceProvider()),
      ],
      child: SmartTutorApp(),
    ),
  );
}

class SmartTutorApp extends StatelessWidget {
  @override
  Widget build(BuildContext context) {
    return MaterialApp(
      title: 'Smart Tutor',
      theme: ThemeData(
        primarySwatch: Colors.blue,
        visualDensity: VisualDensity.adaptivePlatformDensity,
        useMaterial3: true,
      ),
      home: SplashScreen(),
      debugShowCheckedModeBanner: false,
    );
  }
}
\end{lstlisting}

\section{Sentiment Hub Implementation}

\subsection{UI Component}
\begin{lstlisting}[style=sqlstyle, caption=Sentiment Screen UI]
import 'package:flutter/material.dart';
import 'package:provider/provider.dart';
import 'package:smart_tutor/models/message.dart';
import 'package:smart_tutor/providers/sentiment_provider.dart';
import 'package:smart_tutor/widgets/chat_bubble.dart';
import 'package:smart_tutor/widgets/message_input.dart';

class SentimentScreen extends StatelessWidget {
  final int lessonId;
  
  SentimentScreen({required this.lessonId});
  
  @override
  Widget build(BuildContext context) {
    return Scaffold(
      appBar: AppBar(
        title: Text("Academic Sentiment Hub"),
        elevation: 2,
        actions: [
          IconButton(
            icon: Icon(Icons.refresh),
            onPressed: () => context.read<SentimentProvider>().fetchMessages(lessonId),
          ),
        ],
      ),
      body: Column(
        children: [
          // Messages List
          Expanded(
            child: Consumer<SentimentProvider>(
              builder: (context, provider, child) {
                if (provider.isLoading) {
                  return Center(child: CircularProgressIndicator());
                }
                
                if (provider.messages.isEmpty) {
                  return Center(
                    child: Column(
                      mainAxisAlignment: MainAxisAlignment.center,
                      children: [
                        Icon(Icons.forum, size: 64, color: Colors.grey),
                        SizedBox(height: 16),
                        Text(
                          "No messages yet",
                          style: TextStyle(color: Colors.grey),
                        ),
                        Text(
                          "Be the first to ask a question!",
                          style: TextStyle(color: Colors.grey),
                        ),
                      ],
                    ),
                  );
                }
                
                return RefreshIndicator(
                  onRefresh: () => provider.fetchMessages(lessonId),
                  child: ListView.builder(
                    reverse: true,
                    padding: EdgeInsets.all(16),
                    itemCount: provider.messages.length,
                    itemBuilder: (context, index) {
                      final message = provider.messages[index];
                      return ChatBubble(
                        message: message.text,
                        timestamp: message.createdAt,
                        isStudent: message.role == 'student',
                        userName: message.userName,
                      );
                    },
                  ),
                );
              },
            ),
          ),
          
          // Message Input
          MessageInput(
            onSend: (text) {
              context.read<SentimentProvider>().sendMessage(
                text: text,
                lessonId: lessonId,
                userId: context.read<AuthProvider>().userId!,
              );
            },
          ),
        ],
      ),
    );
  }
}
\end{lstlisting}

\section{Admin Resource Management}

\subsection{PDF Management Screen}
\begin{lstlisting}[style=sqlstyle, caption=Resource Management UI]
import 'package:flutter/material.dart';
import 'package:provider/provider.dart';
import 'package:smart_tutor/models/resource.dart';
import 'package:smart_tutor/providers/resource_provider.dart';
import 'package:smart_tutor/widgets/resource_card.dart';
import 'package:file_picker/file_picker.dart';

class ResourceManagementScreen extends StatefulWidget {
  @override
  _ResourceManagementScreenState createState() => _ResourceManagementScreenState();
}

class _ResourceManagementScreenState extends State<ResourceManagementScreen> {
  @override
  void initState() {
    super.initState();
    WidgetsBinding.instance.addPostFrameCallback((_) {
      context.read<ResourceProvider>().fetchResources();
    });
  }
  
  Future<void> _uploadPDF() async {
    FilePickerResult? result = await FilePicker.platform.pickFiles(
      type: FileType.custom,
      allowedExtensions: ['pdf'],
    );
    
    if (result != null) {
      final file = result.files.first;
      await context.read<ResourceProvider>().uploadResource(
        filePath: file.path!,
        fileName: file.name,
        lessonId: selectedLessonId,
      );
    }
  }
  
  @override
  Widget build(BuildContext context) {
    return Scaffold(
      appBar: AppBar(
        title: Text("PDF Resource Management"),
        actions: [
          IconButton(
            icon: Icon(Icons.upload_file),
            onPressed: _uploadPDF,
            tooltip: "Upload PDF",
          ),
        ],
      ),
      body: Consumer<ResourceProvider>(
        builder: (context, provider, child) {
          if (provider.isLoading) {
            return Center(child: CircularProgressIndicator());
          }
          
          return GridView.builder(
            padding: EdgeInsets.all(16),
            gridDelegate: SliverGridDelegateWithFixedCrossAxisCount(
              crossAxisCount: 2,
              crossAxisSpacing: 16,
              mainAxisSpacing: 16,
              childAspectRatio: 0.8,
            ),
            itemCount: provider.resources.length,
            itemBuilder: (context, index) {
              final resource = provider.resources[index];
              return ResourceCard(
                resource: resource,
                onDelete: () => _confirmDelete(resource),
                onView: () => _viewResource(resource),
              );
            },
          );
        },
      ),
    );
  }
  
  void _confirmDelete(Resource resource) {
    showDialog(
      context: context,
      builder: (context) => AlertDialog(
        title: Text("Delete Resource"),
        content: Text("Are you sure you want to delete '${resource.originalFilename}'?"),
        actions: [
          TextButton(
            onPressed: () => Navigator.pop(context),
            child: Text("Cancel"),
          ),
          TextButton(
            onPressed: () {
              context.read<ResourceProvider>().deleteResource(resource.id);
              Navigator.pop(context);
            },
            child: Text("Delete", style: TextStyle(color: Colors.red)),
          ),
        ],
      ),
    );
  }
}
\end{lstlisting}

% ============================================
% CHAPTER 8: BACKEND IMPLEMENTATION
% ============================================
\chapter{Backend Implementation: Node.js \& Express}
\label{ch:backend}

\section{Database Configuration}
\begin{lstlisting}[style=sqlstyle, caption=Database Connection Pool (db.js)]
const mysql = require('mysql2');
require('dotenv').config();

// Create connection pool for better performance
const pool = mysql.createPool({
  host: process.env.DB_HOST || 'localhost',
  user: process.env.DB_USER || 'smart_tutor_admin',
  password: process.env.DB_PASSWORD || 'securepassword123',
  database: process.env.DB_NAME || 'smart_tutor',
  port: process.env.DB_PORT || 3306,
  
  // Connection pool settings
  waitForConnections: true,
  connectionLimit: 20, // Maximum concurrent connections
  queueLimit: 0, // Unlimited queue for waiting connections
  enableKeepAlive: true,
  keepAliveInitialDelay: 0,
  
  // Timeout settings
  connectTimeout: 10000, // 10 seconds
  acquireTimeout: 10000, // 10 seconds
  
  // SSL configuration for production
  ssl: process.env.NODE_ENV === 'production' 
    ? { rejectUnauthorized: true }
    : undefined
});

// Promise wrapper for async/await support
const promisePool = pool.promise();

// Test connection on startup
promisePool.getConnection()
  .then(connection => {
    console.log('✅ Database connected successfully');
    connection.release();
  })
  .catch(err => {
    console.error('❌ Database connection failed:', err.message);
    process.exit(1);
  });

module.exports = promisePool;
\end{lstlisting}

\section{Authentication Middleware}
\begin{lstlisting}[style=sqlstyle, caption=JWT Authentication Middleware (authMiddleware.js)]
const jwt = require('jsonwebtoken');
const db = require('../config/db');

const authenticateToken = async (req, res, next) => {
  const authHeader = req.headers['authorization'];
  const token = authHeader && authHeader.split(' ')[1];
  
  if (!token) {
    return res.status(401).json({ 
      error: 'Access denied. No token provided.' 
    });
  }
  
  try {
    const verified = jwt.verify(token, process.env.JWT_SECRET);
    req.user = verified;
    
    // Verify user still exists and is active
    const [users] = await db.execute(
      'SELECT id, email, role, status FROM users WHERE id = ? AND status = "active"',
      [verified.userId]
    );
    
    if (users.length === 0) {
      return res.status(401).json({ 
        error: 'User account not found or inactive' 
      });
    }
    
    req.userDetails = users[0];
    next();
  } catch (err) {
    return res.status(403).json({ 
      error: 'Invalid or expired token' 
    });
  }
};

const authorizeRoles = (...allowedRoles) => {
  return (req, res, next) => {
    if (!req.userDetails) {
      return res.status(401).json({ 
        error: 'User details not found' 
      });
    }
    
    if (!allowedRoles.includes(req.userDetails.role)) {
      return res.status(403).json({ 
        error: 'Insufficient permissions' 
      });
    }
    
    next();
  };
};

module.exports = { authenticateToken, authorizeRoles };
\end{lstlisting}

\section{Resource Management Controller}
\begin{lstlisting}[style=sqlstyle, caption=PDF Resource Controller (resourceController.js)]
const fs = require('fs').promises;
const path = require('path');
const { v4: uuidv4 } = require('uuid');
const db = require('../config/db');
const { authenticateToken, authorizeRoles } = require('../middleware/auth');

class ResourceController {
  // Upload PDF resource
  static async uploadResource(req, res) {
    try {
      if (!req.file) {
        return res.status(400).json({ error: 'No file uploaded' });
      }
      
      const { lessonId, difficulty = 'intermediate' } = req.body;
      const userId = req.userDetails.id;
      
      // Generate UUID for filename
      const fileUuid = uuidv4();
      const fileExtension = path.extname(req.file.originalname);
      const newFilename = `${fileUuid}${fileExtension}`;
      
      // Define storage path
      const uploadDir = path.join(__dirname, '../../storage/resources');
      const filePath = path.join(uploadDir, newFilename);
      
      // Ensure directory exists
      await fs.mkdir(uploadDir, { recursive: true });
      
      // Move uploaded file
      await fs.rename(req.file.path, filePath);
      
      // Store metadata in database
      const [result] = await db.execute(
        `INSERT INTO resources 
         (file_uuid, original_filename, file_path, file_size, mime_type, 
          lesson_id, difficulty, uploader_id) 
         VALUES (?, ?, ?, ?, ?, ?, ?, ?)`,
        [
          fileUuid,
          req.file.originalname,
          `/storage/resources/${newFilename}`,
          req.file.size,
          req.file.mimetype,
          lessonId || null,
          difficulty,
          userId
        ]
      );
      
      res.status(201).json({
        message: 'Resource uploaded successfully',
        resourceId: result.insertId,
        fileUuid: fileUuid
      });
      
    } catch (error) {
      console.error('Upload error:', error);
      res.status(500).json({ 
        error: 'Failed to upload resource',
        details: error.message 
      });
    }
  }
  
  // Delete PDF resource
  static async deleteResource(req, res) {
    const { resourceId } = req.params;
    const userId = req.userDetails.id;
    
    try {
      // Start transaction
      await db.execute('START TRANSACTION');
      
      // 1. Get resource details
      const [resources] = await db.execute(
        `SELECT file_uuid, file_path, uploader_id 
         FROM resources WHERE id = ?`,
        [resourceId]
      );
      
      if (resources.length === 0) {
        await db.execute('ROLLBACK');
        return res.status(404).json({ error: 'Resource not found' });
      }
      
      const resource = resources[0];
      
      // 2. Verify ownership or admin rights
      if (resource.uploader_id !== userId && req.userDetails.role !== 'admin') {
        await db.execute('ROLLBACK');
        return res.status(403).json({ 
          error: 'Not authorized to delete this resource' 
        });
      }
      
      // 3. Delete from database
      await db.execute(
        'DELETE FROM resources WHERE id = ?',
        [resourceId]
      );
      
      // 4. Delete physical file
      const fullPath = path.join(__dirname, '../..', resource.file_path);
      try {
        await fs.unlink(fullPath);
        console.log(`Deleted file: ${fullPath}`);
      } catch (fsError) {
        console.warn('File deletion warning:', fsError.message);
        // Continue even if file not found - database record is already deleted
      }
      
      // 5. Commit transaction
      await db.execute('COMMIT');
      
      res.status(200).json({ 
        message: 'Resource deleted successfully' 
      });
      
    } catch (error) {
      await db.execute('ROLLBACK');
      console.error('Deletion error:', error);
      res.status(500).json({ 
        error: 'Failed to delete resource',
        details: error.message 
      });
    }
  }
  
  // Get all resources
  static async getResources(req, res) {
    try {
      const { lessonId, difficulty, page = 1, limit = 20 } = req.query;
      const offset = (page - 1) * limit;
      
      let query = `
        SELECT r.*, u.email as uploader_email, l.title as lesson_title
        FROM resources r
        LEFT JOIN users u ON r.uploader_id = u.id
        LEFT JOIN lessons l ON r.lesson_id = l.id
        WHERE r.is_active = TRUE
      `;
      const params = [];
      
      if (lessonId) {
        query += ' AND r.lesson_id = ?';
        params.push(lessonId);
      }
      
      if (difficulty) {
        query += ' AND r.difficulty = ?';
        params.push(difficulty);
      }
      
      query += ' ORDER BY r.upload_date DESC LIMIT ? OFFSET ?';
      params.push(parseInt(limit), parseInt(offset));
      
      const [resources] = await db.execute(query, params);
      
      // Get total count for pagination
      const [countResult] = await db.execute(
        'SELECT COUNT(*) as total FROM resources WHERE is_active = TRUE'
      );
      
      res.json({
        resources,
        pagination: {
          page: parseInt(page),
          limit: parseInt(limit),
          total: countResult[0].total,
          totalPages: Math.ceil(countResult[0].total / limit)
        }
      });
      
    } catch (error) {
      console.error('Get resources error:', error);
      res.status(500).json({ 
        error: 'Failed to fetch resources',
        details: error.message 
      });
    }
  }
}

module.exports = ResourceController;
\end{lstlisting}

\section{Sentiment Hub Controller}
\begin{lstlisting}[style=sqlstyle, caption=Sentiment Controller (sentimentController.js)]
const db = require('../config/db');

class SentimentController {
  // Post new sentiment message
  static async postMessage(req, res) {
    try {
      const { lessonId, message, sentimentType = 'question', priority = 'medium' } = req.body;
      const userId = req.userDetails.id;
      
      if (!message || message.trim().length === 0) {
        return res.status(400).json({ error: 'Message cannot be empty' });
      }
      
      if (message.length > 1000) {
        return res.status(400).json({ error: 'Message too long (max 1000 characters)' });
      }
      
      // Insert message
      const [result] = await db.execute(
        `INSERT INTO sentiment_logs 
         (user_id, lesson_id, message, sentiment_type, priority) 
         VALUES (?, ?, ?, ?, ?)`,
        [userId, lessonId, message.trim(), sentimentType, priority]
      );
      
      // Get the complete message with user details
      const [messages] = await db.execute(
        `SELECT sl.*, u.email, CONCAT(u.first_name, ' ', u.last_name) as user_name, u.role
         FROM sentiment_logs sl
         JOIN users u ON sl.user_id = u.id
         WHERE sl.id = ?`,
        [result.insertId]
      );
      
      res.status(201).json({
        message: 'Sentiment posted successfully',
        data: messages[0]
      });
      
    } catch (error) {
      console.error('Post message error:', error);
      res.status(500).json({ 
        error: 'Failed to post message',
        details: error.message 
      });
    }
  }
  
  // Get messages for a lesson
  static async getMessages(req, res) {
    try {
      const { lessonId } = req.params;
      const { page = 1, limit = 50, status } = req.query;
      const offset = (page - 1) * limit;
      
      let query = `
        SELECT sl.*, u.email, CONCAT(u.first_name, ' ', u.last_name) as user_name, u.role
        FROM sentiment_logs sl
        JOIN users u ON sl.user_id = u.id
        WHERE sl.lesson_id = ?
      `;
      const params = [lessonId];
      
      if (status) {
        query += ' AND sl.status = ?';
        params.push(status);
      }
      
      query += ' ORDER BY sl.created_at DESC LIMIT ? OFFSET ?';
      params.push(parseInt(limit), parseInt(offset));
      
      const [messages] = await db.execute(query, params);
      
      // Get total count
      const [countResult] = await db.execute(
        'SELECT COUNT(*) as total FROM sentiment_logs WHERE lesson_id = ?',
        [lessonId]
      );
      
      res.json({
        messages,
        pagination: {
          page: parseInt(page),
          limit: parseInt(limit),
          total: countResult[0].total,
          totalPages: Math.ceil(countResult[0].total / limit)
        }
      });
      
    } catch (error) {
      console.error('Get messages error:', error);
      res.status(500).json({ 
        error: 'Failed to fetch messages',
        details: error.message 
      });
    }
  }
  
  // Update message status (for admin/teacher responses)
  static async updateMessageStatus(req, res) {
    try {
      const { messageId } = req.params;
      const { status, response } = req.body;
      const userId = req.userDetails.id;
      
      // Check if user is teacher or admin
      if (!['teacher', 'admin'].includes(req.userDetails.role)) {
        return res.status(403).json({ 
          error: 'Only teachers and admins can update message status' 
        });
      }
      
      const [result] = await db.execute(
        `UPDATE sentiment_logs 
         SET status = ?, updated_at = CURRENT_TIMESTAMP 
         WHERE id = ?`,
        [status, messageId]
      );
      
      if (result.affectedRows === 0) {
        return res.status(404).json({ error: 'Message not found' });
      }
      
      // If response provided, log it as a separate message
      if (response) {
        await db.execute(
          `INSERT INTO sentiment_logs 
           (user_id, lesson_id, message, sentiment_type, priority, status, is_response_to) 
           VALUES (?, 
           (SELECT lesson_id FROM sentiment_logs WHERE id = ?), 
           ?, 'response', 'medium', 'answered', ?)`,
          [userId, messageId, response, messageId]
        );
      }
      
      res.json({ 
        message: 'Message status updated successfully' 
      });
      
    } catch (error) {
      console.error('Update message error:', error);
      res.status(500).json({ 
        error: 'Failed to update message',
        details: error.message 
      });
    }
  }
}

module.exports = SentimentController;
\end{lstlisting}

\section{API Routes Configuration}
\begin{lstlisting}[style=sqlstyle, caption=Express Routes Configuration (routes.js)]
const express = require('express');
const router = express.Router();
const multer = require('multer');
const { authenticateToken, authorizeRoles } = require('./middleware/auth');
const AuthController = require('./controllers/authController');
const ResourceController = require('./controllers/resourceController');
const SentimentController = require('./controllers/sentimentController');

// Configure multer for file uploads
const upload = multer({
  dest: 'uploads/temp/',
  limits: {
    fileSize: 50 * 1024 * 1024, // 50MB limit
  },
  fileFilter: (req, file, cb) => {
    if (file.mimetype === 'application/pdf') {
      cb(null, true);
    } else {
      cb(new Error('Only PDF files are allowed'), false);
    }
  }
});

// Public routes
router.post('/auth/register', AuthController.register);
router.post('/auth/login', AuthController.login);
router.post('/auth/refresh', AuthController.refreshToken);

// Protected routes
router.use(authenticateToken);

// User routes
router.get('/users/profile', AuthController.getProfile);
router.put('/users/profile', AuthController.updateProfile);

// Resource routes
router.get('/resources', ResourceController.getResources);
router.get('/resources/:resourceId', ResourceController.getResource);
router.post('/resources/upload', 
  authorizeRoles('admin', 'teacher'), 
  upload.single('file'), 
  ResourceController.uploadResource
);
router.delete('/resources/:resourceId', 
  authorizeRoles('admin', 'teacher'), 
  ResourceController.deleteResource
);

// Sentiment routes
router.post('/sentiment', SentimentController.postMessage);
router.get('/sentiment/lesson/:lessonId', SentimentController.getMessages);
router.put('/sentiment/:messageId/status', 
  authorizeRoles('admin', 'teacher'), 
  SentimentController.updateMessageStatus
);

// Admin-only routes
router.get('/admin/users', 
  authorizeRoles('admin'), 
  AuthController.getAllUsers
);
router.put('/admin/users/:userId/role', 
  authorizeRoles('admin'), 
  AuthController.updateUserRole
);

// Error handling middleware
router.use((err, req, res, next) => {
  console.error('Route error:', err);
  
  if (err instanceof multer.MulterError) {
    return res.status(400).json({ 
      error: 'File upload error', 
      details: err.message 
    });
  }
  
  res.status(err.status || 500).json({
    error: err.message || 'Internal server error'
  });
});

module.exports = router;
\end{lstlisting}

% ============================================
% CHAPTER 9: AGILE PROJECT MANAGEMENT
% ============================================
\chapter{Agile Implementation: Scrum Framework}
\label{ch:agile}

\section{Project Methodology}
The Smart Tutor project was developed using the Scrum agile framework, with three 2-week sprints focusing on core authentication, sentiment engine development, and admin resource tools.

\section{Product Backlog}
\begin{longtable}{|p{2cm}|p{6cm}|p{3cm}|p{2cm}|}
\hline
\textbf{ID} & \textbf{User Story} & \textbf{Priority} & \textbf{Status} \\ \hline
PB-01 & As a user, I want to sign up and log in securely & High & Completed \\ \hline
PB-02 & As a student, I want to post academic inquiries so teachers can help me & High & Completed \\ \hline
PB-03 & As a teacher, I want to respond to student inquiries in real-time & High & Completed \\ \hline
PB-04 & As an admin, I want to upload PDF study materials to the VPS & Medium & Completed \\ \hline
PB-05 & As an admin, I want to delete outdated PDFs to keep the library clean & Medium & Completed \\ \hline
PB-06 & As a user, I want to upload a local profile picture & Low & Completed \\ \hline
PB-07 & As a user, I want dark mode for night-time studying & Low & Completed \\ \hline
PB-08 & As a student, I want to search through past sentiments & Medium & In Progress \\ \hline
PB-09 & As a teacher, I want analytics on student confusion patterns & Low & Planned \\ \hline
PB-10 & As an admin, I want to batch upload multiple PDFs & Low & Planned \\ \hline
\end{longtable}

\section{Sprint Execution}

\subsection{Sprint 1: Core Authentication \& Infrastructure (Days 1-14)}
\textbf{Goal:} Establish backend API and basic user access system.

\begin{longtable}{|p{3cm}|p{10cm}|p{2cm}|}
\hline
\textbf{Task} & \textbf{Description} & \textbf{Status} \\ \hline
INF-01 & Setup Ubuntu VPS with Nginx and MySQL & Completed \\ \hline
INF-02 & Implement JWT-based authentication in Node.js & Completed \\ \hline
INF-03 & Create Flutter Login and Signup UI & Completed \\ \hline
INF-04 & Integrate local SQLite for profile storage & Completed \\ \hline
INF-05 & Configure HTTPS with Let's Encrypt & Completed \\ \hline
\end{longtable}

\subsection{Sprint 2: Sentiment Hub Development (Days 15-28)}
\textbf{Goal:} Enable real-time student-teacher communication.

\begin{longtable}{|p{3cm}|p{10cm}|p{2cm}|}
\hline
\textbf{Task} & \textbf{Description} & \textbf{Status} \\ \hline
SENT-01 & Create Sentiment database schema & Completed \\ \hline
SENT-02 & Develop Flutter Sentiment Screen UI & Completed \\ \hline
SENT-03 & Implement API polling for real-time updates & Completed \\ \hline
SENT-04 & Add lesson context tagging system & Completed \\ \hline
SENT-05 & Implement message notifications & Completed \\ \hline
\end{longtable}

\subsection{Sprint 3: Admin Resource Management (Days 29-42)}
\textbf{Goal:} Empower admins with comprehensive PDF management.

\begin{longtable}{|p{3cm}|p{10cm}|p{2cm}|}
\hline
\textbf{Task} & \textbf{Description} & \textbf{Status} \\ \hline
RES-01 & Implement file upload with multer & Completed \\ \hline
RES-02 & Create admin PDF management dashboard & Completed \\ \hline
RES-03 & Script secure file deletion process & Completed \\ \hline
RES-04 & Finalize DevOps deployment with PM2 & Completed \\ \hline
RES-05 & Implement resource usage analytics & Completed \\ \hline
\end{longtable}

\section{Sprint Review Metrics}
\begin{longtable}{|p{4cm}|p{3cm}|p{3cm}|p{3cm}|}
\hline
\textbf{Metric} & \textbf{Sprint 1} & \textbf{Sprint 2} & \textbf{Sprint 3} \\ \hline
Story Points Completed & 28 & 32 & 30 \\ \hline
Velocity & 28 & 30 & 29 \\ \hline
Bugs Reported & 12 & 8 & 5 \\ \hline
Test Coverage & 75\% & 82\% & 88\% \\ \hline
Team Satisfaction & 4.2/5 & 4.5/5 & 4.7/5 \\ \hline
\end{longtable}

% ============================================
% CHAPTER 10: DEVOPS AND DEPLOYMENT
% ============================================
\chapter{DevOps and Deployment}
\label{ch:devops}

\section{Infrastructure Stack}
\begin{longtable}{|p{4cm}|p{10cm}|}
\hline
\textbf{Component} & \textbf{Technology \& Configuration} \\ \hline
Operating System & Ubuntu 22.04 LTS (Long Term Support) \\ \hline
Web Server & Nginx 1.18 with reverse proxy configuration \\ \hline
Process Manager & PM2 5.2 for process monitoring and auto-restart \\ \hline
Database & MySQL 8.0 with InnoDB engine and utf8mb4 encoding \\ \hline
Backend Runtime & Node.js 18.x LTS with Express.js 4.x \\ \hline
Frontend Framework & Flutter 3.x with Dart 3.x \\ \hline
File Storage & Local VPS storage with RAID 1 configuration \\ \hline
SSL Certificate & Let's Encrypt with auto-renewal via Certbot \\ \hline
Monitoring & PM2 monitoring + Custom logging to Elastic Stack \\ \hline
Backup Solution & Automated daily backups to AWS S3 \\ \hline
\end{longtable}

\section{Deployment Architecture}
\begin{figure}[H]
    \centering
    \includegraphics[width=0.9\textwidth]{deployment-architecture.png}
    \caption{Production Deployment Architecture}
    \label{fig:deployment}
\end{figure}

\section{Deployment Scripts}

\subsection{Production Deployment Script}
\begin{lstlisting}[style=sqlstyle, caption=Production Deployment Script (deploy.sh)]
#!/bin/bash

# Smart Tutor Production Deployment Script
# Usage: ./deploy.sh [environment]

set -e  # Exit on error

ENVIRONMENT=${1:-production}
TIMESTAMP=$(date +%Y%m%d_%H%M%S)
BACKUP_DIR="/backups/smart_tutor/$TIMESTAMP"

echo "�� Starting deployment to $ENVIRONMENT environment"
echo "�� Timestamp: $TIMESTAMP"

# Step 1: Backup current deployment
echo "�� Creating backup..."
mkdir -p $BACKUP_DIR
systemctl stop smart-tutor-backend 2>/dev/null || true
cp -r /var/www/smart-tutor/* $BACKUP_DIR/ 2>/dev/null || true
echo "✅ Backup created at: $BACKUP_DIR"

# Step 2: Pull latest code
echo "�� Pulling latest code..."
cd /var/www/smart-tutor
git fetch origin main
git reset --hard origin/main

# Step 3: Install backend dependencies
echo "�� Installing backend dependencies..."
cd backend
npm ci --only=production
npm audit fix

# Step 4: Run database migrations
echo "��️ Running database migrations..."
if [ -f "migrations/latest.sql" ]; then
    mysql -u smart_tutor_admin -p$DB_PASSWORD smart_tutor < migrations/latest.sql
fi

# Step 5: Build frontend
echo "��️ Building Flutter frontend..."
cd ../frontend
flutter clean
flutter pub get
if [ "$ENVIRONMENT" = "production" ]; then
    flutter build apk --release --split-per-abi
    flutter build ios --release --no-codesign
else
    flutter build apk --debug
fi

# Step 6: Restart services
echo "�� Restarting services..."
cd ../backend
pm2 delete smart-tutor-api 2>/dev/null || true
pm2 start ecosystem.config.js --env $ENVIRONMENT
pm2 save

# Step 7: Update Nginx configuration
echo "�� Updating Nginx..."
nginx -t
systemctl reload nginx

# Step 8: Cleanup old backups (keep last 7 days)
echo "�� Cleaning up old backups..."
find /backups/smart_tutor/* -type d -ctime +7 -exec rm -rf {} \;

echo "�� Deployment completed successfully!"
echo "�� Service status:"
pm2 status smart-tutor-api
\end{lstlisting}

\subsection{PM2 Ecosystem Configuration}
\begin{lstlisting}[style=sqlstyle, caption=PM2 Configuration (ecosystem.config.js)]
module.exports = {
  apps: [{
    name: 'smart-tutor-api',
    script: './backend/server.js',
    instances: 'max', // Use all CPU cores
    exec_mode: 'cluster', // Cluster mode for load balancing
    autorestart: true,
    watch: false,
    max_memory_restart: '1G',
    
    // Environment variables
    env: {
      NODE_ENV: 'development',
      PORT: 3000,
      JWT_SECRET: 'dev-secret-change-in-production',
      DB_HOST: 'localhost'
    },
    env_production: {
      NODE_ENV: 'production',
      PORT: 3000,
      JWT_SECRET: process.env.JWT_SECRET,
      DB_HOST: process.env.DB_HOST,
      LOG_LEVEL: 'info'
    },
    
    // Logging configuration
    log_date_format: 'YYYY-MM-DD HH:mm:ss',
    error_file: '/var/log/smart-tutor/api-error.log',
    out_file: '/var/log/smart-tutor/api-out.log',
    
    // Monitoring
    min_uptime: '60s',
    max_restarts: 10,
    
    // Graceful shutdown
    kill_timeout: 5000,
    
    // Node.js arguments
    node_args: '--max-old-space-size=1024'
  }]
};
\end{lstlisting}

\section{Security Protocols}

\subsection{Role-Based Access Control (RBAC)}
\begin{itemize}
    \item \textbf{Student Role:} Read access to resources, write access to sentiment hub
    \item \textbf{Teacher Role:} All student privileges + ability to respond to sentiments
    \item \textbf{Admin Role:} Full system access including user management and resource control
\end{itemize}

\subsection{Security Measures Implemented}
\begin{longtable}{|p{4cm}|p{10cm}|}
\hline
\textbf{Security Aspect} & \textbf{Implementation} \\ \hline
Authentication & JWT tokens with 24-hour expiry and refresh tokens \\ \hline
Password Storage & BCrypt hashing with salt rounds = 12 \\ \hline
Input Validation & Server-side validation using express-validator \\ \hline
SQL Injection Prevention & Parameterized queries via mysql2 library \\ \hline
XSS Protection & Content sanitization and CSP headers \\ \hline
File Upload Security & File type validation, size limits, virus scanning \\ \hline
API Rate Limiting & 100 requests/minute per IP address \\ \hline
HTTPS Enforcement & HSTS headers and SSL redirect \\ \hline
CORS Configuration & Restrictive origin policy for mobile apps \\ \hline
\end{longtable}

\section{Monitoring and Maintenance}

\subsection{Health Check Endpoints}
\begin{lstlisting}[style=sqlstyle, caption=Health Check Implementation]
// Health check endpoint
router.get('/health', async (req, res) => {
  const health = {
    status: 'healthy',
    timestamp: new Date().toISOString(),
    uptime: process.uptime(),
    services: {}
  };
  
  // Check database connection
  try {
    await db.execute('SELECT 1');
    health.services.database = 'healthy';
  } catch (error) {
    health.services.database = 'unhealthy';
    health.status = 'degraded';
  }
  
  // Check file system
  try {
    await fs.access('/var/www/smart-tutor/storage');
    health.services.filesystem = 'healthy';
  } catch (error) {
    health.services.filesystem = 'unhealthy';
    health.status = 'degraded';
  }
  
  res.status(health.status === 'healthy' ? 200 : 503).json(health);
});

// Metrics endpoint (for Prometheus)
router.get('/metrics', (req, res) => {
  const metrics = {
    active_users: activeConnections,
    requests_per_minute: requestCounter.get(),
    memory_usage: process.memoryUsage(),
    cpu_usage: process.cpuUsage()
  };
  res.json(metrics);
});
\end{lstlisting}

\subsection{Backup Strategy}
\begin{itemize}
    \item \textbf{Database:} Daily automated backups with point-in-time recovery
    \item \textbf{Files:} Incremental backups of uploaded resources
    \item \textbf{Configuration:} Version-controlled in Git repository
    \item \textbf{Retention:} 30 days local, 1 year in AWS S3 Glacier
\end{itemize}

% ============================================
% CHAPTER 11: TESTING STRATEGY
% ============================================
\chapter{Testing Strategy}
\label{ch:testing}

\section{Test Pyramid Implementation}
\begin{figure}[H]
    \centering
    \includegraphics[width=0.6\textwidth]{test-pyramid.png}
    \caption{Testing Strategy Pyramid}
    \label{fig:test-pyramid}
\end{figure}

\section{Unit Testing}
\subsection{Backend Unit Tests}
\begin{lstlisting}[style=sqlstyle, caption=Example Unit Test (authController.test.js)]
const request = require('supertest');
const app = require('../app');
const db = require('../config/db');

describe('Authentication Controller', () => {
  beforeAll(async () => {
    // Setup test database
    await db.execute('DELETE FROM users WHERE email LIKE "test%@example.com"');
  });
  
  afterAll(async () => {
    // Cleanup
    await db.end();
  });
  
  describe('POST /auth/register', () => {
    it('should register a new user successfully', async () => {
      const response = await request(app)
        .post('/auth/register')
        .send({
          email: 'test@example.com',
          password: 'SecurePass123!',
          firstName: 'Test',
          lastName: 'User',
          role: 'student'
        });
      
      expect(response.status).toBe(201);
      expect(response.body).toHaveProperty('token');
      expect(response.body.user).toHaveProperty('email', 'test@example.com');
    });
    
    it('should reject duplicate email', async () => {
      const response = await request(app)
        .post('/auth/register')
        .send({
          email: 'test@example.com',
          password: 'AnotherPass123!',
          firstName: 'Duplicate',
          lastName: 'User'
        });
      
      expect(response.status).toBe(400);
      expect(response.body.error).toContain('already exists');
    });
  });
});
\end{lstlisting}

\section{Integration Testing}
\begin{lstlisting}[style=sqlstyle, caption=Integration Test Example]
describe('Resource Management Integration', () => {
  let authToken;
  
  beforeAll(async () => {
    // Login to get token
    const loginRes = await request(app)
      .post('/auth/login')
      .send({ email: 'admin@example.com', password: 'AdminPass123!' });
    
    authToken = loginRes.body.token;
  });
  
  it('should upload and delete PDF resource', async () => {
    // Upload PDF
    const uploadRes = await request(app)
      .post('/resources/upload')
      .set('Authorization', `Bearer ${authToken}`)
      .attach('file', 'test.pdf')
      .field('lessonId', 1)
      .field('difficulty', 'beginner');
    
    expect(uploadRes.status).toBe(201);
    const resourceId = uploadRes.body.resourceId;
    
    // Verify resource exists
    const getRes = await request(app)
      .get(`/resources/${resourceId}`)
      .set('Authorization', `Bearer ${authToken}`);
    
    expect(getRes.status).toBe(200);
    
    // Delete resource
    const deleteRes = await request(app)
      .delete(`/resources/${resourceId}`)
      .set('Authorization', `Bearer ${authToken}`);
    
    expect(deleteRes.status).toBe(200);
    
    // Verify deletion
    const verifyRes = await request(app)
      .get(`/resources/${resourceId}`)
      .set('Authorization', `Bearer ${authToken}`);
    
    expect(verifyRes.status).toBe(404);
  });
});
\end{lstlisting}

\section{Test Coverage Results}
\begin{longtable}{|p{4cm}|p{3cm}|p{3cm}|p{3cm}|}
\hline
\textbf{Component} & \textbf{Line Coverage} & \textbf{Branch Coverage} & \textbf{Function Coverage} \\ \hline
Authentication & 95\% & 92\% & 96\% \\ \hline
Resource Management & 91\% & 88\% & 93\% \\ \hline
Sentiment Hub & 89\% & 85\% & 91\% \\ \hline
User Management & 93\% & 90\% & 94\% \\ \hline
\textbf{Overall} & \textbf{92\%} & \textbf{89\%} & \textbf{93\%} \\ \hline
\end{longtable}

% ============================================
% CHAPTER 12: CONCLUSION AND FUTURE WORK
% ============================================
\chapter{Conclusion and Future Enhancements}
\label{ch:conclusion}

\section{Project Achievements}
The Smart Tutor project successfully delivered a comprehensive educational platform with the following key achievements:

\begin{itemize}
    \item \textbf{Real-time Communication:} Implemented a functional Sentiment Hub enabling direct student-teacher interaction
    \item \textbf{Robust Resource Management:} Built a secure PDF management system with VPS storage integration
    \item \textbf{Scalable Architecture:} Established a three-tier architecture supporting future growth
    \item \textbf{Agile Delivery:} Completed all high-priority features within the 6-week timeline using Scrum
    \item \textbf{Quality Assurance:} Achieved 92\% test coverage with comprehensive testing strategies
\end{itemize}

\section{Technical Strengths}
\begin{longtable}{|p{4cm}|p{10cm}|}
\hline
\textbf{Area} & \textbf{Strength} \\ \hline
Architecture & Clean separation of concerns with three-tier design \\ \hline
Security & Comprehensive security measures including RBAC and input validation \\ \hline
Performance & Optimized database queries and connection pooling \\ \hline
Maintainability & Adherence to SOLID principles and clean code practices \\ \hline
DevOps & Automated deployment and monitoring with PM2 \\ \hline
\end{longtable}

\section{Future Enhancements}

\subsection{Short-term Roadmap (Next 3 Months)}
\begin{longtable}{|p{3cm}|p{10cm}|p{2cm}|}
\hline
\textbf{Feature} & \textbf{Description} & \textbf{Priority} \\ \hline
WebSocket Support & Replace HTTP polling with WebSockets for real-time updates & High \\ \hline
Advanced Search & Full-text search across resources and sentiments & High \\ \hline
Mobile Notifications & Push notifications for new responses & Medium \\ \hline
Analytics Dashboard & Visual insights into student engagement patterns & Medium \\ \hline
Bulk Operations & Batch upload and management of resources & Low \\ \hline
\end{longtable}

\subsection{Long-term Vision (Next 12 Months)}
\begin{itemize}
    \item \textbf{AI-Powered Recommendations:} Machine learning to suggest resources based on student confusion patterns
    \item \textbf{Multi-language Support:} Internationalization for global reach
    \item \textbf{Advanced Analytics:} Predictive analytics for identifying at-risk students
    \item \textbf{Mobile Offline Mode:} Enhanced offline capabilities for low-connectivity areas
    \item \textbf{Integration Ecosystem:} APIs for integration with existing LMS platforms
\end{itemize}

\section{Lessons Learned}
\begin{itemize}
    \item \textbf{Technical:} The importance of proper connection pooling for database performance
    \item \textbf{Process:} Value of comprehensive testing in early sprints to reduce technical debt
    \item \textbf{Team:} Benefits of daily stand-ups for identifying blockers early
    \item \textbf{Product:} User feedback emphasized the need for simpler navigation flows
\end{itemize}

\section{Final Remarks}
The Smart Tutor project demonstrates the successful implementation of modern software engineering practices in an educational technology context. By combining Flutter's reactive UI capabilities with Node.js's efficient backend processing and secure VPS storage management, the platform provides a robust foundation for enhancing student-teacher collaboration. The Scrum framework proved effective in managing priorities and delivering value incrementally, while the technical architecture ensures scalability and maintainability for future enhancements.

The platform stands as a testament to the power of focused, user-centered design combined with solid technical execution, ready to make a meaningful impact on educational outcomes through technology-enabled collaboration.

\end{document}